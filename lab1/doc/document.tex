\documentclass[12pt, a4paper]{article}
\usepackage[english,russian]{babel}
\usepackage{csquotes}
\usepackage{graphicx}

\oddsidemargin=-0.4mm
\textwidth=160mm
\topmargin=4.6mm
\textheight=210mm
\parindent=0pt
\parskip=3pt

\begin{document}
	
	\begin{titlepage}
		
		\begin{center}
			
			\bfseries
			{\Large Московский авиационный институт\\ 
				(национальный исследовательский университет)}
			
			\vspace{48pt}
			{\large Факультет информационных технологий и прикладной 
				математики}
			
			\vspace{36pt}
			{\large Кафедра вычислительной математики и программирования}
			
			\vspace{48pt}
			Лабораторная работа \textnumero 1 по курсу 
			\enquote{Компьютерная графика}
			
		\end{center}
		
		\vspace{72pt}
		\begin{flushright}
			\begin{tabular}{rl}
				Студент: & Я.\,А. Борисов \\
				Преподаватель: & Л.\,Н. Чернышов \\
				Группа: & М8О-308Б-20 \\
				Дата: & \\
				Оценка: & \\
				Подпись: & \\
			\end{tabular}
		\end{flushright}
		
		\vfill
		
		\begin{center}
			
			\bfseries
			Москва, \the\year
			
		\end{center}
		
	\end{titlepage}
	
	\pagebreak
	
	\section*{Лабораторная работа \textnumero 1}
	
	\par\textbf{Тема: } {
		Построение изображений 2D-кривых.
	}
	
	\par\textbf{Задача: } {
		Написать и отладить программу, строящую изображение заданной 
		замечательной кривой.
	}
	
	\par\textbf{Вариант \textnumero 3: }
	$x=a*\cos^{3} \varphi,  
	y=a*\sin^{3} \varphi
	
	\section{Решение}
	Для выполнения поставленной задачи было принято решение использовать язык 
	программирования Python и его модули matplotlib (для отрисовки графика и 
	координатных осей), numpy и sympy (для построения массива значения функций от параметра $t$). Из модуля numpy пригодилась функция linspace, с помощью
	которой получили массив $T$ размером 1000 равномерно распределённых 
	чисел в интервале от $-5$ до $5$. Размер массива был выбран так, 
	чтобы график функции был построен с приемлемой точностью. Из модуля 
	matplotlib использовались методы axhline, axvline и arrow для построения 
	координатных осей, а также функция plot для отрисовки графика. Значения функции при каждом значении $t$ из $T$ были высчитаны с помощью модуля sympy и записаны в массив для построения графика.  Полученный резльтат выводится на экран с 
	помощью функции show. Результат работы программы можно увидеть ниже.
	
	\includegraphics[scale=0.75]{Figure_1.png}
	
	\pagebreak
	
	\section{Выводы}
	Проделав лабораторную работу, познакомился с отрисовкой 2D-изображений, 
	отрисовал двумерную систему координат и график, а также укрепил навыки 
	работы с matplotlib, sympy и numpy.
	
\end{document}